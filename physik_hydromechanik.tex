%
% $Id: physik_hydromechanik.tex,v 1.1 2003/10/24 15:42:53 ninja Exp $
%

\section{Definitionen}

\subsection{Hydrostatik}
Lehre vom Kr"aftegleichgewicht in ruhenden Fl"ussigkeiten.

\subsection{Hydrodynamik}
Lehre vom Str"omungsgesetzen in bewegten Fl"ussigkeiten.

\subsection{Dichte $\rho$}
\begin{equation}
	\rho=\frac{\text{Masse}}{\text{Volumen}}=\frac{\partial M}{\partial V}\unit{\frac{kg}{m^3}}
\end{equation}

\subsection{Spezifisches Gewicht $\gamma$}
\begin{equation}
	\gamma=\frac{\text{Gewicht}}{\text{Volumen}}=\frac{\partial G}{\partial V}=\frac{\partial m\cdot g}{\partial V}\unit{\frac{N}{m^3}}
\end{equation}
\begin{equation}
	\gamma=g\cdot\rho
\end{equation}

\subsection{Druck $p$}
\begin{equation}
	p=\frac{\text{Kraft}}{\text{Fl"ache}}=\frac{\partial F_N}{\partial A}
\end{equation}

\subsubsection{Einheiten des Druckes}
\begin{align*}
	1 Pa &= 1\frac{N}{m^2}\qquad\text{(Pascal)} \\
	1 bar &= 10^5 Pa \\
	1 atm &= 1.01325 bar\qquad\text{(Normaldruck auf Meeresh"ohe)} \\
		&= 760mm Hg \\
	1 torr &= 1mm Hg\quad (0^{\circ}C) \\
		&= 1.3332\cdot 10^2 Pa \\
	1 at &= 10m H_2O\quad ( \text{bei } 4^{\circ}C) \\
		&= \frac{kp}{cm^2}
\end{align*}

\section{Druck}

\subsection{Schweredruck}
\begin{center}
	\begin{pspicture}(-0.5,-0.5)(3,1.5)
		\psframe[fillstyle=hlines*,hatchcolor=blue,fillcolor=white,linestyle=none](0,0)(3,1)
		\psline[linecolor=blue]{-}(0,1)(3,1)
		\psline[linewidth=1.5pt]{-}(0,1.5)(0,0)(3,0)(3,1.5)
		\rput[b](1.5,1.2){$p_0$}
		\rput*[B](1.5,0.5){$\rho$}
		\psline{|->}(-0.3,1.0)(-0.3,-0.3)\rput[rt](0,-0.2){$z$}
	\end{pspicture}
\end{center}
\begin{equation}
	p_{(z)}=\frac{\partial F}{\partial A}=p_0+\rho\cdot g\cdot z
\end{equation}

\subsection{Hydrostatische "Ubersetzung}
\begin{center}
	\begin{pspicture}(-0.5,0)(4,3)
		\psframe[fillstyle=hlines*,hatchcolor=blue,fillcolor=white,linestyle=none](0,0)(3,1.3)
		\psframe[fillstyle=solid,fillcolor=white,linestyle=none](1.5,1)(3,1.5)
		\psframe[fillstyle=solid,fillcolor=white,linestyle=none](0.5,0.5)(1.5,2.0)
		\psline[linecolor=blue](1.5,1)(3,1)
		\psline[linecolor=blue](0,1.3)(0.5,1.3)
		\psline[linewidth=1.5pt](0,1.5)(0,0)(3,0)(3,1.5)
		\psline[linewidth=1.5pt](0.5,1.5)(0.5,0.5)(1.5,0.5)(1.5,1.5)
		\rput[r](-0.2,1.6){$A_1$}
		\rput[l](3.2,1.3){$A_2$}
		\rput[b](1.0,1.0){$p_0$}
		\pcline{->}(0.25,2.5)(0.25,1.5)\Aput{$F_1$}
		\pcline{->}(2.25,2.5)(2.25,1.5)\Aput{$F_2$}
		\rput*[b](2.25,0.3){$\rho$}
		\pcline{|-|}(-0.2,1.3)(-0.2,0)\Bput{$h_1$}
		\pcline{|-|}(3.2,1.0)(3.2,0)\Aput{$h_2$}
	\end{pspicture}
\end{center}
\begin{equation}
	\frac{A_2}{A_1}=\frac{F_2}{F_1}+\frac{A_2}{A_1}\cdot\rho\cdot g\cdot(h_2-h_1)
\end{equation}

\subsection{Auftrieb $F_A$ (Archimedes)}
Der Auftrieb $F_A$ ist dem Betrag nach gleich dem Gewicht der verdr"angten Fl"ussigkeit. Die Luft kann eigentlich vernachl"assigt werden.
\begin{itemize}
	\item $\overrightarrow{F}_A$ ist parallel zum vorhandenen Beschleunigungsfeld (im Allgemeinen $\overrightarrow{g}$).
	\item Angriffspunkt von $F_A$ ist der Schwerpunkt der \textbf{verdr"angten Fl"ussigkeit}.
\end{itemize}

\subsection{Bernoulli}
\begin{center}
	\begin{pspicture}(-1,-1)(4,1.2)
		\psline{-}(0, 0.5)(1, 0.5)(2, 1)(3, 1)
		\psline{-}(0,-0.5)(1,-0.5)(2,-1)(3,-1)
		\psellipse[fillstyle=hlines*,hatchcolor=lightgray,fillcolor=white](0.25,0)(0.2,0.5)
		\psellipse[fillstyle=hlines*,hatchcolor=lightgray,fillcolor=white](2.75,0)(0.2,1.0)
		\pcline{->}(-1,0)(0,0)\Aput{\small $\overrightarrow{v}_1$}
		\pcline{->}(3,0)(4,0)\Aput{\small $\overrightarrow{v}_2$}
		\rput[b](0.25,0.8){$A_1$}
		\rput[l](2.9,0.8){$A_2$}
	\end{pspicture}
\end{center}
\begin{equation}
	v_1\cdot A_1=v_2\cdot A_2\qquad\Longleftrightarrow\qquad\frac{v_1}{v_2}=\frac{A_2}{A_1}
\end{equation}
\begin{equation}
	p_1+\frac{1}{2}\cdot\rho\cdot v_1^2+\rho\cdot h_1\cdot g = p_2+\frac{1}{2}\cdot\rho\cdot v_2^2+ \rho\cdot h_2\cdot g
\end{equation}
\noindent Gilt nur f"ur eine Stromlinie einer inkompressiblen Fl"ussigkeit.\\
\noindent Erkl"arung:
\begin{equation*}
	\underbrace{p_1}_{\text{\small Betriebsdruck}}+\underbrace{\frac{1}{2}\cdot\rho\cdot v^2}_{\text{\small dyn. Druck, Staudruck}}+\underbrace{\rho\cdot h\cdot g}_{\text{\small Schweredruck}}
\end{equation*}
\noindent wobei: {\em Betriebsdruck}$+${\em Schweredruck}$=${\em Statischer Druck}

\subsection{Ausflussgeschwindigkeit}
\begin{center}
	\begin{pspicture}(-0.5,0)(4,3)
		\psframe[fillstyle=hlines*,hatchcolor=blue,fillcolor=white,linestyle=none](0,0)(3,2)
		\psframe[fillstyle=solid,fillcolor=white,linestyle=none](2.5,0.5)(3.1,2.1)
		\psline[linecolor=blue](0,2)(2.5,2)
		\psline[linewidth=1.5pt](0,2.5)(0,0)(3,0)
		\psline[linewidth=1.5pt](2.5,2.5)(2.5,0.5)(3.0,0.5)
		\rput[l](2.7,2.0){$A_1$}
		\rput[br](3.0,0.7){$A_2$}
		\pcline{->}(1.0,3.0)(1.0,2.0)\Aput{$v_1$}
		\pcline{->}(3,0.25)(4,0.25)\Aput{$v_2$}
		\pcline{|-|}(-0.3,2.0)(-0.3,0.0)\Bput{$h$}
	\end{pspicture}
\end{center}
\begin{equation}
	\frac{A_1}{A_2}>>1\qquad\text{das heisst}\quad v_1\rightarrow 0\qquad\Longrightarrow\qquad v_2=\sqrt{2\cdot g\cdot h}
\end{equation}

\subsection{Pitot-Rohr}
\begin{center}
	\begin{pspicture}(-1,-1.5)(4,1.5)
		\psframe[fillstyle=hlines*,hatchcolor=blue,fillcolor=white,linestyle=none](1.5,-1.5)(3.0,-0.75)
		\psframe[fillstyle=solid,fillcolor=white,linestyle=none](1.75,-1.25)(2.75,-0.7)
		\psframe[fillstyle=solid,fillcolor=white,linestyle=none](2.75,-1.1)(3,-0.7)
		\psline[linecolor=blue](1.5,-0.75)(1.75,-0.75)
		\psline[linecolor=blue](2.75,-1.1)(3,-1.1)
		\psline{-}(0.0,0.25)(3.0,  0.25)(3.0, -1.5)(1.5, -1.5)(1.5, -0.5)(0.5, -0.5)(0.0, -0.25)(2.75,-0.25)(2.75,-1.25)(1.75,-1.25)(1.75,-0.5)(2.5, -0.5)(2.5, -0.25)
		\psline{-}(2.5,0.25)(2.5,0.5)(0.5,0.5)(0,0.25)
		\psline[linecolor=lightgray]{-}(1.6,-1.25)(0.7,-1.25)
		\rput[r](0.6,-1.25){$\rho_F$}
		\psline[linecolor=lightgray]{-}(2.8,0.1)(2.8,0.5)
		\rput[b](2.8,0.6){$\rho_L$}
		\pcline{->}(-1,0)(0,0)\Aput{$v$}
		\psline[linecolor=gray](1.75,-0.75)(3.5,-0.75)
		\psline[linecolor=gray](3,-1.1)(3.5,-1.1)
		\pcline{|-|}(3.5,-0.75)(3.5,-1.1)\Aput{$\Delta h$}
	\end{pspicture}
\end{center}
\begin{equation}
	v=\sqrt{\frac{2\cdot g\cdot \rho_F\cdot\Delta h}{\rho_L}}
\end{equation}

\section{Widerstand}

\subsection{Reibungswiderstand}
Der Reibungswiderstand is proportional zur Geschwindigkeit.
\begin{equation}
	F_r=6\cdot\pi\cdot r\cdot v\cdot\eta
\end{equation}
\begin{gather*}
	\eta\unit{N\cdot\frac{s}{m}=\frac{kg}{m\cdot s}}\qquad\text{: Viskosit"at} \\
	\eta_{\text{Luft}} = 1.8\cdot 10^{-5}\unit{\frac{kg}{m\cdot s}}
\end{gather*}

\subsubsection{Beispiel: Regentropfen}
$v_0$ : station"are Fallgeschwindigkeit
\begin{equation*}
	v_0=\frac{2\cdot r^2\cdot g\cdot(\rho_{\text{Wasser}}-\rho_{\text{Luft}})}{9\cdot\eta_{\text{Luft}}}
\end{equation*}

\subsection{Druckwiderstand}
Dominant bei turbulenter Str"omung
\begin{equation}
	F_D= c_D\cdot A\cdot\frac{1}{2}\cdot\rho\cdot v^2
\end{equation}
\begin{align*}
	A\qquad&\text{: Querschnitt bez"uglich Bewegungsrichtung} \\
	c_D\qquad&\text{: Druckwiderstandsbeiwert, Fummelfaktor} \\
	\rho\qquad&\text{: Dichte des Mediums}
\end{align*}

\subsection{Gesamtwiderstand}
\begin{equation}
	F_W=F_r+F_D\qquad\Longrightarrow\qquad F_W=c_W\cdot A\cdot\frac{1}{2}\cdot\rho\cdot v^2
\end{equation}
\noindent $c_W$ : Widerstandsbeiwert, Abh"angig von der Geschwindigkeit. Konstant wenn $F_r << F_D$
\begin{center}
	\begin{pspicture}(-1.5,-0.5)(6.5,1)
		\pcline{->}(-1.5,0.5)(-0.5,0.5)\Aput{$v$}
		\psline{-}(0,0)(0,1)\rput[t](0,-0.2){1.17}
		\psarc[fillstyle=hlines*](1,0.5){0.5}{90}{270}\psline(1,1)(1,0)\rput[t](1,-0.2){0.42}
		\psarc[fillstyle=hlines*](2,0.5){0.5}{270}{90}\psline(2,1)(2,0)\rput[t](2,-0.2){1.17}
		\pscircle[fillstyle=hlines*](4,0.5){0.5}\rput[t](4,-0.2){0.47}
		\pscurve(6,0.5)(5,0.75)(5,0.25)(6,0.5)\rput[t](5.5,-0.2){0.04}
	\end{pspicture} \\
	(gelten f"ur Reynolds $\approx 7\cdot 10^4$)
\end{center}

\subsection{Reynod'sche Zahl}
\begin{equation}
	Re=\frac{\hat{l}\cdot\rho\cdot v}{\eta}
\end{equation}
\noindent mit
\begin{align*}
	\hat{l}=\frac{4\cdot A}{u}\qquad &\text{: charakteristische Gr"osse, hydrodynamischer Durchmesser} \\
	u\qquad &\text{: Umfang}
\end{align*}
\noindent $Re_{krtitisch} \approx 2300$

%
% EOF
%
