%
% $Id: mathematik_diskretemathematik.tex,v 1.5 2003/10/18 21:38:33 ninja Exp $
%

\newcommand{\ztf}{{\huge\mathfrak{z}}}

\section{Begriffe}

\paragraph{Aussage ``$p$''}
\begin{align*}
	\text{Formal:}&\quad p \\
	\mathcal{D}\text{:} & \quad\text{true/false}, \text{wahr/falsch}, 1/0
\end{align*}

\paragraph{Aussage ``\textit{nicht $p$}''}
\begin{align*}
	\text{Formal:}&\quad\urcorner p\qquad\text{(Negation)} \\
	\mathcal{D}\text{:} & \quad\text{false/true}, \text{falsch/wahr}, 0/1
\end{align*}

\paragraph{Aussage ``\textit{$P$ und $Q$}''}
\begin{gather*}
	\text{Formal:} \quad p\wedge q\quad\text{oder}\quad p\cdot q\qquad\text{(Konjunktion)}
\end{gather*}
\begin{center}
	\begin{tabular}{cc|c}
		\hline
		$p$ & $q$ & $p\wedge q$ \\
		\hline
		$0$ & $0$ & $0$ \\
		$0$ & $1$ & $0$ \\
		$1$ & $0$ & $0$ \\
		$1$ & $1$ & $1$ \\
	\end{tabular}
\end{center}

\paragraph{Aussage ``\textit{$P$ oder $Q$}''}
\begin{gather*}
	\text{Formal:} \quad p\vee q\quad\text{oder}\quad p+q\qquad\text{(Disjunktion)}
\end{gather*}
\begin{center}
	\begin{tabular}{cc|c}
		\hline
		$p$ & $q$ & $p\vee q$ \\
		\hline
		$0$ & $0$ & $0$ \\
		$0$ & $1$ & $1$ \\
		$1$ & $0$ & $1$ \\
		$1$ & $1$ & $1$ \\
	\end{tabular}
\end{center}

\paragraph{Aussage ``\textit{wenn $P$ dann $Q$}''}
\begin{gather*}
	\text{Formal:} \quad p\rightarrow q\qquad\text{(Konditional)}
\end{gather*}
\begin{center}
	\begin{tabular}{cc|c}
		\hline
		$p$ & $q$ & $p\vee q$ \\
		\hline
		$0$ & $0$ & $1$ \\
		$0$ & $1$ & $1$ \\
		$1$ & $0$ & $0$ \\
		$1$ & $1$ & $1$ \\
	\end{tabular}
\end{center}

\paragraph{Aussage ``\textit{$Q$ genau dann wenn $P$}''}
\begin{gather*}
	\text{Formal:} \quad p\leftrightarrow q\qquad\text{(Bikonditional)}
\end{gather*}
\begin{center}
	\begin{tabular}{cc|c}
		\hline
		$p$ & $q$ & $p\vee q$ \\
		\hline
		$0$ & $0$ & $1$ \\
		$0$ & $1$ & $0$ \\
		$1$ & $0$ & $0$ \\
		$1$ & $1$ & $1$ \\
	\end{tabular}
\end{center}

\paragraph{Aussage ``\textit{aus $P$ folgt $Q$}''}
\begin{gather*}
	\text{Formal:} \quad P\Longrightarrow Q\qquad\text{(Folgerung)}
\end{gather*}

\paragraph{Aussage ``\textit{$P$ ist "aquivalent zu $Q$}''}
\begin{gather*}
	\text{Formal:} \quad P\Longleftrightarrow Q\qquad\text{("Aquivalenz)}
\end{gather*}

\subsection{Allgemein}
\begin{align*}
	W,F,P,Q,\ldots\qquad &\hat{=}\text{Atomare Bausteine, Atome, Formeln} \\
	W,F\qquad &\hat{=}\text{Wahrheitswerte}
\end{align*}

\subsection{Priorit"aten}
\begin{center}
	\begin{tabular}{c|cc}
		& \multicolumn{2}{c}{$\underrightarrow{\text{gleichbleibend}}$} \\
		\hline
		\multirow{3}{5mm}{\begin{sideways}$\underleftarrow{\text{abnehmend}}$\end{sideways}} & & \\
			& $\urcorner$ & \\
			& $\wedge$ & $\vee$ \\
			& $\rightarrow$ & $\leftrightarrow$ \\
	\end{tabular}
\end{center}
\vspace{2mm}

\subsection{Allgemeing"ultigkeit}
Eine Formel ist allgemeing"ultig, wenn sie f"ur jede Belegung der Wahrheitswert $W$
ergibt.

Auch {\em Tautologie} genannt.

\subsection{Definitionen}
Eine Formel heisst erf"ullbar, wenn es min. eine Belegung gibt f"ur die
sie Wahr ist.

Eine Formel $B$ folgt aus einer anderen Formel $A$, wenn f"ur jede Belegung
die $A$ Wahr macht, $B$ wahr wird.

Eine Formelmenge $y$ folgt aus der Formelmenge $x$, wenn jede Formel aus $y$
Wahr wird f"ur Belegungen die $x$ Wahr macht.

Zwei Formeln oder Formelmengen sind zueinander "aquivalent, wenn sie
gegenseitig auseinander Folgern.

\section{Beweis-Strategien}
\subsection{M"ogliche Strategien}
\begin{itemize}
\item Beweis durch {\em Kontraposition}
\item Beweis durch {\em Widerspruch}
\item Beweis durch {\em Fallunterscheidung}
\item {\em Direkter Beweis}
\item {\em Induktionsbeweis}
\end{itemize}

\subsection{Widerspr"uchlichkeit}
Eine Formal $A$ heisst {\em widerspr"uchlich}, wenn sowohl $A$ als auch
$\urcorner A$ f"ur gleiche Belegungen Wahr werden.

Eine Formelmenge heisst Widerspr"uchlich, wenn f"ur min. eine Formel obiges gilt.

\section{Entscheidungsverfahren}
Entscheidungsverfahren f"ur Erf"ullbarkeit entspricht einem {\em Algroithmus}

Erf"ullbarkeit, Allgemeing"ultigkeit und logische Folgerungen f"ur endliche Formelmengen
sind entscheidbar.

Umfang: Anzahl Zeilen der Wahrheitstabelle = $2^\text{Anzahl Atome}$

\section{Normalformen}
Ein {\em Literal} ($L$) entspricht einem aussagekr"aftigen Atom oder seiner Verneinung.

\subsection{Konjunktive Normalform (KNF)}
Formal: Produkt von Summen
\begin{equation}
	\bigwedge\limits_{j=1}^n\left({\bigvee\limits_{i=1}^m L_{ij}}\right)
\end{equation}
\begin{itemize}
\item Sind nicht eindeutig bestimmt!
\item Verschiedene Formeln k"onnen die gleiche KNF besitzen
\end{itemize}

\subsection{Disjunktive Normalform (DNF)}
Formal: Summe von Produkten
\begin{equation}
	\bigvee\limits_{j=1}^n\left({\bigwedge\limits_{i=1}^m L_{ij}}\right)
\end{equation}

\subsection{Ausgezeichnete Normalform}
Alle Literale kommen in einem Disjunktionsterm bzw. Konjunktionsterm vor. Die
ausgezeichnete Normalform ist nicht die k"urzeste M"oglichkeit eine Formel zu beschreiben.
\begin{itemize}
\item \textbf{Ausgezeichnete KNF:} Nullwerte (Falsch) in der Wahrheitstabelle betrachten
\item \textbf{Ausgezeichnete DNF:} Einswerte (Wahr) in der Wahrheitstabelle betrachten
\end{itemize}

\section{Ableitungen}
\begin{gather}
	\underbrace{A\quad\wedge\quad\urcorner A}_\text{Pr"amisse}\quad\Longrightarrow\quad\underbrace{F}_\text{Konklusion}
\end{gather}
\vspace{5mm}
\begin{center}
	\begin{rotate}{180}
		\setlength{\GapWidth}{2cm}
		\setlength{\GapDepth}{1cm}
		\begin{bundle}{\begin{rotate}{180}Konklusion\end{rotate}}
		\chunk{\begin{rotate}{180}Pr"amisse\end{rotate}}
		\chunk{\begin{rotate}{180}Pr"amisse\end{rotate}}
		\chunk{\begin{rotate}{180}Pr"amisse\end{rotate}}
		\end{bundle}
	\end{rotate}
\end{center}
\vspace{5mm}

\section{Boolsche Algebra}
Grundmenge: $\mathbb{B}=\{0,1\}$
\subsection{Axiome}
\begin{align}
	0\cdot 0 &= 1\cdot 0=0\cdot 1=0 \\
	0+1 &= 1+0 = 1+1 = 0 \\
	0+0 &= 0 \\
	1\cdot 1 &= 1 \\
	\overline{0} &= 1\qquad\text{Komplement, Negation} \\
	\overline{1} &= 0\qquad\text{Komplement, Negation}
\end{align}

\subsection{Dualisieren}
\begin{equation*}
	s^d\quad\hat{=}\quad\text{``$s$ dual''}
\end{equation*}
Inverseion der Funktionen und Operationen:
\begin{align*}
	+\quad & \rightarrow\quad\cdot \\
	\cdot\quad & \rightarrow\quad + \\
	0 \quad & \rightarrow\quad 1 \\
	1 \quad & \rightarrow\quad 0
\end{align*}

\subsection{Normalformen}
\subsubsection{Disjuktive Normalform}
\begin{center}
{\em Summe von Produkten}
\end{center}
Siehe auch {\em disjunktive Normalform} in der Aussagenlogik.

\subsubsection{Konjunktive Normalform}
\begin{center}
{\em Produkt von Summen}
\end{center}
Siehe auch {\em konjunktive Normalform} in der Aussagenlogik.

\section{Mengen}
\subsection{Begriffe}
\subsubsection{Mengeh und Teilmengen}
\begin{center}
	\begin{pspicture}(-2,-2)(2,2)
		\pscircle(0,0){1.5}
		\pscircle[linecolor=blue](0,0){1.25}
		\pscircle[linecolor=red](0,0){1.0}
		\rput[Bl](1.7,1.5){$G$}
		\rput[Bl](1.7,1.0){$B$}
		\rput[Bl](1.7,0.5){$A$}
		\psline{-}(1.7,1.5)(1.05,1.05)
		\psline{-}(1.7,1.0)(1.0, 0.7)
		\psline{-}(1.7,0.5)(0.95,0.3)
	\end{pspicture}
\end{center}
\begin{equation}
	\mathbb{A}\subset\mathbb{B}
\end{equation}
\begin{center}
	\begin{pspicture}(-2,-2)(2,2)
		\pscircle(0,0){1.5}
		\pscircle[linecolor=blue](0,0){1.0}
		\pscircle[linecolor=red,linestyle=dashed](0,0){1.0}
		\rput[Bl](1.7,1.5){$G$}
		\rput[Bl](1.7,1.0){$B$}
		\rput[Bl](1.7,0.5){$A$}
		\psline{-}(1.7,1.5)(1.05,1.05)
		\psline{-}(1.7,1.0)(0.95,0.3)
		\psline{-}(1.7,0.5)(0.95,0.3)
	\end{pspicture}
\end{center}
\begin{equation}
	\mathbb{A}\subseteq\mathbb{B}\quad\text{und}\quad\mathbb{B}\subseteq\mathbb{A}\qquad\Longrightarrow\quad\mathbb{A}=\mathbb{B}
\end{equation}

\subsubsection{Leere Menge}
\begin{equation}
	\text{Leere Menge: }\qquad\emptyset\quad\text{oder}\quad\{\}
\end{equation}

\subsubsection{Standardmengen}
\begin{center}
	\begin{tabular}{ll}
		$\mathbb{N}, \mathbb{N}_0$		& Nat"urliche Zahlen \\
		$\mathbb{Z}^+, \mathbb{Z}_0^+, \mathbb{Z}^-, \mathbb{Z}_0^-, \mathbb{Z}$	& Nat"urliche ganze Zahlen \\
		$\mathbb{Q}^+, \mathbb{Q}_0^+, \mathbb{Q}^-, \mathbb{Q}_0^-, \mathbb{Q}$	& \\
		$\mathbb{R}^+, \mathbb{R}_0^+, \mathbb{R}^-, \mathbb{R}_0^-, \mathbb{R}$	& reelle Zahlen \\
		$\mathbb{C}$					& komplexe Zahlen
	\end{tabular}
\end{center}

\subsubsection{Potenzmenge}
$\mathbb{P}(\mathbb{A})$ entspricht der Potenzmenge von $\mathbb{A}$. Die Potenzmenge
ist die Menge aller Teilmengen, inkl. $\emptyset$.

\subsection{Verkn"upfungen}
\begin{center}
	Vereinigung: $\mathbb{A}\cup\mathbb{B}$ \\ %% A or B
	\begin{pspicture}(0,0)(5,3)
		\rput[Br](1.0,2.0){$A$}
		\rput[Bl](4.0,2.0){$B$}
		\psarc[linewidth=2pt,linecolor=red,fillcolor=white,fillstyle=hlines,hatchcolor=red](2.0,1.5){1.0}{60}{-60}
		\psarc[linewidth=2pt,linecolor=red,fillcolor=white,fillstyle=hlines,hatchcolor=red](3.0,1.5){1.0}{240}{120}
	\end{pspicture}
\end{center}
\begin{center}
	Schnittmenge: $\mathbb{A}\cap\mathbb{B}$ \\ %% A and B
	\begin{pspicture}(0,0)(5,3)
		\rput[Br](1.0,2.0){$A$}
		\rput[Bl](4.0,2.0){$B$}
		\pscircle(2,1.5){1.0}
		\pscircle(3,1.5){1.0}
		\psarc[linewidth=2pt,linecolor=red,fillcolor=white,fillstyle=hlines,hatchcolor=red](2.0,1.5){1.0}{-60}{60}
		\psarc[linewidth=2pt,linecolor=red,fillcolor=white,fillstyle=hlines,hatchcolor=red](3.0,1.5){1.0}{120}{240}
	\end{pspicture}
\end{center}
\begin{center}
	Differenz: $\mathbb{A}\backslash\mathbb{B}$ \\ %% A minus B
	\begin{pspicture}(0,0)(5,3)
		\rput[Br](1.0,2.0){$A$}
		\rput[Bl](4.0,2.0){$B$}
		\pscircle[linewidth=2pt,linecolor=red,fillcolor=white,fillstyle=hlines,hatchcolor=red](2,1.5){1.0}
		\pscircle[fillcolor=white,fillstyle=solid](3,1.5){1.0}
		\psarc[linewidth=2pt,linecolor=red](3.0,1.5){1.0}{120}{240}
	\end{pspicture}
\end{center}
\begin{center}
	Symmetrische Differenz: $\mathbb{A}\triangle\mathbb{B}$ \\ %% A symminus B
	\begin{pspicture}(0,0)(5,3)
		\rput[Br](1.0,2.0){$A$}
		\rput[Bl](4.0,2.0){$B$}
		\psarc[linewidth=2pt,linecolor=red,fillcolor=white,fillstyle=hlines,hatchcolor=red](2.0,1.5){1.0}{60}{-60}
		\psarc[linewidth=2pt,linecolor=red,fillcolor=white,fillstyle=hlines,hatchcolor=red](3.0,1.5){1.0}{240}{120}
		\psarc[linewidth=2pt,linecolor=red,fillcolor=white,fillstyle=solid](2.0,1.5){1.0}{-60}{60}
		\psarc[linewidth=2pt,linecolor=red,fillcolor=white,fillstyle=solid](3.0,1.5){1.0}{120}{240}
	\end{pspicture}
\end{center}

\subsection{Dualisieren}
\begin{center}\begin{tabular}{c c}
	Ersetzen 		& durch \\
	\hline
	$\cup$			& $\cap$ \\
	$\cap$			& $\cup$ \\
	$\emptyset$		& $\mathbb{G}$ \\
	$\mathbb{G}$	& $\emptyset$ \\
\end{tabular}\end{center}
\begin{gather*}
	(\mathbb{A}\cap\mathbb{B})\cup\emptyset=\mathbb{D}\quad\Longrightarrow\quad\mathbb{D}^d=(\mathbb{A}\cup\mathbb{B})\cap\mathbb{G} \\
	\mathbb{A}=\mathbb{B}\Longleftrightarrow\mathbb{A}^d=\mathbb{B}^d
\end{gather*}

\subsection{Venn-Diagramm}
Grafische Veranschaulichung von Mengen. Alternative zu Venn-Diagramm: Zugeh"origkeitstabelle.
\begin{center}
	\begin{pspicture}(0,0)(5,3)
		\rput[Br](1.0,2.0){$A$}
		\rput[Bl](4.0,2.0){$B$}
		\pscircle(2,1.5){1.0}
		\pscircle(3,1.5){1.0}
		\psarc[linewidth=2pt,linecolor=red,fillcolor=white,fillstyle=hlines,hatchcolor=red](2.0,1.5){1.0}{-60}{60}
		\psarc[linewidth=2pt,linecolor=red,fillcolor=white,fillstyle=hlines,hatchcolor=red](3.0,1.5){1.0}{120}{240}
	\end{pspicture} \\
	$\mathbb{A}\cap\mathbb{B}$
\end{center}

\subsection{Indexmenge}
Zusammenfassung aller Werte, die ein Index annhemen kann.
\begin{equation}
	\overline{\bigcup\limits_{i\in\text{II}}\mathbb{A}_i}=\bigcap\limits_{i\in\text{II}}\mathbb{A}_i\qquad\qquad
	\overline{\bigcap\limits_{i\in\text{II}}\mathbb{A}_i}=\bigcup\limits_{i\in\text{II}}\mathbb{A}_i
\end{equation}

\subsection{M"achtigkeit}
$|\mathbb{A}|\quad\hat{=}$ Anzahl der Elemente von $\mathbb{A}$, sofern $\mathbb{A}$ endlich. \\
\noindent $card(\mathbb{A})\quad\hat{=}$ M"achtigkeit der Menge $\mathbb{A}$. \\
\noindent F"ur endliche Mengen: $card(\mathbb{A})=|\mathbb{A}|$. F"ur unendliche Mengen dient $\mathbb{N}$ als Referenzmenge.
\begin{align*}
	|\mathbb{A}\cup\mathbb{B}| &= |\mathbb{A}|+|\mathbb{B}|-|\mathbb{A}\cap\mathbb{B}| \\
	|\mathbb{A}\cup\mathbb{B}\cup\mathbb{C}| &= |\mathbb{A}|+|\mathbb{B}|+|\mathbb{C}|-|\mathbb{A}\cap\mathbb{B}|-|\mathbb{A}\cap\mathbb{C}|-|\mathbb{B}\cap\mathbb{C}|+|\mathbb{A}\cap\mathbb{B}\cap\mathbb{C}|
\end{align*}

\section{Wahrscheinlichkeitstheorie}
\begin{equation}
	P=\frac{\text{Anzahl g"unstige F"alle}}{\text{Anzahl m"ogliche F"alle}}
\end{equation}
\begin{equation*}
	P(\mathbb{A}) = p\qquad\text{mit}\quad p\in [0,1]
\end{equation*}

\section{Kombinatorik}
\subsection{Permutation}
Die Permutation ist die Anordnung von $n$ Elementen {\em ohne Wiederholung} und {\em mit Reihenfolge}.
\begin{equation}
	P_n=n!
\end{equation}

\subsection{Variationen}
Variationen ist die Anordnung von $k$ Elementen aus insgesammt $n$ {\em mit Reihenfolge}.
\begin{equation}
	V_n^k=\frac{n!}{(n-k)!}
\end{equation}
\noindent ``Anzahl Variationen von $n$ Elementen zur $k$-ten Klasse''

\subsection{Variationen mit Wiederholung}
Es sind die Anordnung von $k$ aus $n$ Elementen {\em mit Wiederholung} und {\em mit Reihenfolge}.
\begin{equation}
	\phantom{V}^wV_n^k=n^k
\end{equation}

\subsection{Kombinationen}
Kombinationen sind die Anordnung von $k$ aus $n$ Elementen {\em ohne Reihenfolge} und {\em ohne Wiederholung}.
\begin{equation}
	C_n^k=\frac{n!}{k!\cdot(n-k)!}=\binom{n}{k}
\end{equation}

\subsection{Kombinationen mit Wiederholungen}
Es ist die Anordnung von $k$ aus $n$ Elementen {\em mit Wiederholung} und {\em ohne Reihenfolge}.
\begin{equation}
	\phantom{C}^wC_n^k=\binom{n+k-1}{k}
\end{equation}

\subsection{Klassifizierung}
\begin{center}
	\pstree[nodesep=4pt,levelsep=1cm]{\TR{Anordnung}}{
		\psset{linestyle=none}
		\pstree{\TR{Reihenfolge}}{
			\pstree{\TR{Wiederholung}}{
				\TR{Auswahl aus allen}
			}
		}
		\psset{linestyle=solid}
		\pstree{\TR{wesentlich}}{

			\pstree[levelsep=0.5cm]{\TR{mit}}{
				\psset{linestyle=none}
				\TR{$\phantom{V}^wV_n^k$}
			}

			\pstree{\TR{ohne}}{

				\pstree[levelsep=0.5cm]{\TR{ja}}{
				\psset{linestyle=none}
					\TR{$V_n^k$}
				}

				\pstree[levelsep=0.5cm]{\TR{nein}}{
				\psset{linestyle=none}
					\TR{$V_n^n=P_n$}
				}
			}
		}
		\pstree{\TR{unwesentlich}}{

			\pstree[levelsep=0.5cm]{\TR{mit}}{
				\psset{linestyle=none}
				\TR{$\phantom{C}^wC_n^k$}
			}

			\pstree{\TR{ohne}}{

				\pstree[levelsep=0.5cm]{\TR{ja}}{
				\psset{linestyle=none}
					\TR{$C_n^k$}
				}

				\pstree[levelsep=0.5cm]{\TR{nein}}{
				\psset{linestyle=none}
					\TR{$C_n^n=1$}
				}
			}
		}
		\psset{linestyle=none}
		\TR{\phantom{Auswahl aus allen}}
	}
\end{center}

\begin{align*}
	P_n &= n! \\
	\phantom{V}^wV_n^k &= n^k \\
	V_n^k &= \frac{n!}{(n-k)!} \\
	\phantom{C}^wC_n^k &= \binom{n+k-1}{k} \\
	C_n^k &= \frac{n!}{k!\cdot(n-k)!}=\binom{n}{k}
\end{align*}

\section{Rekursionen}
Eine Rekursion ist eine Beziehung einer Funktion an der Stelle $n$ zu Werten an der Stelle $n-1,n-2,\ldots,n-k$.

\noindent Beispiel: $f_n=f_{n-1}+f_{n-2}$

\section{Erzeugende Funktionen}
Entspricht einer Aufl"osung einer Rekursion, mit welcher direkt das $n$-te Glied berechnet werden kann, anstatt von Anfang bis zum $n$-ten Glied durchzurechnen.

\noindent \underline{Definition:} $a_0,a_1,\ldots$ sei eine Folge von reellen Zahlen, dann heisst die Funktion
\begin{equation*}
	f:x\mapsto a_0+a_1+a_2x^2+\cdots\quad=\sum\limits_{k=0}^\infty a_kx^k
\end{equation*}
\noindent ezeugende Funktion f"ur die gegebene Zahlenfolge.

\section{Differenzengleichungen}
Auch Rekursinosgleichungen genannt.

\subsection{Definition}
Differenzengleichung $k$-ter Ordnung: $F(y_t,y_{t-1},y_{t-2},\ldots,y_{t-k})$

\subsection{Klassierung}
\begin{center}
	\pstree[levelsep=1cm,nodesep=3pt]{\TR{Differenzengleichung}}{
		\TR{nicht linear}
		\pstree{\TR{linear}}{
			\pstree{\TR{1./2.~Ordnung}}{
				\pstree{\TR{\small konst.~Koeff.}}{
					\TR{\small homogen}
					\TR{\small inhomogen}
				}
				\pstree{\TR{\small var.~Koeff.}}{
					\TR{\small homogen}
					\TR{\small inhomogen}
				}
			}
			\pstree{\TR{h"ohere~Ordnung}}{
				\pstree{\TR{\small konst.~Koeff.}}{
					\TR{\small homogen}
					\TR{\small inhomogen}
				}
				\pstree{\TR{\small var.~Koeff.}}{
					\TR{\small homogen}
					\TR{\small inhomogen}
				}
			}
		}
	}
\end{center}

\noindent Der Typ bestimmt die L"osungsmethode. L"osungsmethoden f"ur lineare Differezengleichung mit konstanten Koeffizienten:
\begin{enumerate}
	\item Z-Transformation
	\item Konventionell \begin{enumerate}
			\item homogen \begin{enumerate}
				\item mittels Ansatz
				\end{enumerate}
			\item inhomogen \begin{enumerate}
				\item mittels Ansatz vom Typ der St"orfunktion
				\item erzeugende Funktion
				\item Operator-Technik
				\end{enumerate}
		\end{enumerate}
\end{enumerate}

\subsection{Lineare DGL, konst. Koeff, Homogen}
Standardgleichung:
\begin{equation}
	a_0\cdot y_{t+n}+a_1\cdot y_{t+n-1}+a_2\cdot y_{t+n-2}+\cdots+a_n\cdot y_t =0\quad\forall t
\end{equation}

\noindent Ansatz f"ur L"osung: $y_t=c\cdot\lambda^t$
\begin{align*}
	\Longrightarrow\qquad
		y_{t+1} &= c\cdot\lambda^{t+1} \\
		y_{t+2} &= c\cdot\lambda^{t+2} \\
		y_{t+n} &= c\cdot\lambda^{t+n}
\end{align*}
\noindent eingesetzt in DGL:
\begin{align*}
	a_0\cdot c\cdot\lambda^{t+n}+a_1\cdot c\cdot\lambda^{t+n-1}+\cdots &= 0 \\
	a_0\cdot\lambda^{t+n}+a_1\cdot\lambda^{t+n-1}+\cdots &= 0
\end{align*}
\noindent Dies ist ein Polynom $n$-ten Grades, d.h. $n$ L"osungen: $\lambda_1,\lambda_2,\ldots,\lambda_n$

\subsection{Ansatz vom Typ der St"orfunktion}
\begin{center}\begin{tabular}{l l}
	St"orfunktion & Ansatz f"ur L"osung \\
	\hline
	$\beta^t$						& $A\cdot\beta^t$ \\
	$\sin(\alpha t)$				& $A\cdot\cos(\alpha t)+B\cdot\sin(\alpha t)$ \\
	$\cos(\alpha t)$				& $A\cdot\cos(\alpha t)+B\cdot\sin(\alpha t)$ \\
	$P_m(t)$						& $A_0\cdot t^m+A_1\cdot t^{m-1}+\cdots+A_m$ \\
	$\beta^t\cdot P_m(t)$			& $\beta^t(A_0\cdot t^m+A_1\cdot t^{m-1}+\cdots+A_m)$ \\
	$\beta^t\cdot\sin(\alpha t)$	& $\beta^t\cdot(A\cdot\cos(\alpha t)+B\cdot\sin(\alpha t))$ \\
\end{tabular}\end{center}

\subsection{Unabh"angigkeit der L"osung}
{\em Casorati-Determinante}
\begin{equation}
	\det\begin{pmatrix}
		{(y_{k})}_1		& {(y_{k})}_2		& \cdots	& {(y_{k})}_n \\
		{(y_{k+1})}_1	& {(y_{k+1})}_2		& \cdots	& {(y_{k+1})}_n \\
		\vdots			& \vdots			&			& \vdots \\
		{(y_{k+n-1})}_1	& {(y_{k+n-1})}_2	& \cdots	& {(y_{k+n-1})}_n \\
	\end{pmatrix}\neq 0
\end{equation}

\subsection{Komplexe Nullstellen}
\begin{equation}
	e^{j\phi}=\cos(\phi)+j\sin(\phi)\qquad\phi\in\mathbb{R}
\end{equation}
\begin{equation}
	\Longrightarrow\qquad c_s\cdot|\lambda_s|^k\cdot\cos(\phi\cdot k)+c_{s+1}\cdot|\lambda_s|^k\cdot\cos(\phi\cdot k)+
\end{equation}

\subsection{L"osen mittels erzeugende Funktion}
\begin{equation}
	\{a_k\}\mapsto\sum\limits_{k=0}^\infty a_k\cdot x^k = G(x)
\end{equation}
\noindent Methode:
\begin{enumerate}
	\item Multiplikation der DGL mit $x^k$
	\item Summation der DGLs von $k=0\ldots\infty$
	\item Vereinfachen unter Benutzung von $G(x)$, der erzeugenden Funktion
	\item Isolierung der erzeugenden Funktion
	\item Reihenentwicklung f"ur erzeugende Funktion
\end{enumerate}

\section{Z-Transformation}
\begin{center}\begin{tabular}{c c c c c}
	(1) & Differenzgleichung & $\ztransf$ & gew"ohnliche Gleichung & (2) \\
	& & & $\downarrow$ & \\
	(4) & L"osung der DGL & $\ztransf$ & $\underbrace{\text{L"osung der Gleichung}}_{\text{Bildbereich}}$ & (3) \\
\end{tabular}\end{center}

\subsection{Definition}
\begin{equation}
	\{y_k\}\ztransf Y(z)=\sum\limits_{k=0}^\infty y_k\cdot z^{-k}\qquad z\in\mathbb{C}
\end{equation}
\noindent Kurzschreibweise:
\begin{eqnarray*}
	\ztf\left(\{y_k\}\right) \quad & = & \quad Y(z) \\
	\{y_k\}\quad & \ztransf & \quad Y(z) \\
	\ztf(y_k) \quad & = & \quad Y(z)
\end{eqnarray*}

\subsection{Eigenschaften}

\subsubsection{Linearit"at}
\begin{equation}
	\ztf(ay_k+bx_k)=a\cdot\ztf(y_k)+b\cdot\ztf(x_k)
\end{equation}

\subsubsection{Indexverschiebung}
\begin{align}
	\ztf(y_{k+s}) &= z^s\left({Y_{(z)}-\sum\limits_{k=0}^{s-1}\frac{y^k}{z^k}}\right) \\
	\ztf(y_{k-s}) &= \frac{1}{z^s}\left({Y_{(z)}-\sum\limits_{k=1}^{s}y_{-k}\cdot z^k}\right)
\end{align}

\subsubsection{D"ampfungssatz}
\begin{equation}
	\ztf\left(y_k\cdot e^{-ak}\right) = Y(z\cdot e^a)
\end{equation}

\subsubsection{Differentiation im Bildbereich}
\begin{equation}
	\frac{\partial}{\partial z}Y(z)=-\frac{1}{z}\cdot\ztf(k\cdot y_k)
\end{equation}

\subsubsection{Differentation nach einem Parameter}
\begin{equation}
	\ztf\left(\frac{\partial}{\partial a}y_k(a)\right)=\frac{\partial}{\partial a}\ztf\left(y_k(a)\right)
\end{equation}

\subsubsection{Faltungssatz}
\begin{equation}
	\ztf(y_k)\cdot\ztf(x_k)=\ztf(y_k\ast x_k)
\end{equation}
\paragraph{Faltung}
\begin{equation}
	y_k\ast x_k =\sum\limits_{m=0}^k x_m\cdot y_{k-m}
\end{equation}

\section{Numerik}

\subsection{Definition}
Maschinenzahlen sind die im Rechner {\em exakt} darstellbaren Zahlen.

\begin{verbatim}
    DecimalToBaseB(x,a)    // 0 <= x(10) < 1
      a[0] = 0
      k = 0
      while (x != 0)
        k = k + 1
        a[k] = floor(B*x)
        x = B * x - a[k]
      end while
\end{verbatim}

\subsection{Festpunktzahlen}
\begin{equation}
	z=\sigma\cdot\sum\limits_{i=m}^n a_i\cdot B^i \qquad m,n\in\mathbb{Z}\quad(\text{typ. } m<0<n)
\end{equation}
\begin{equation*}
	\text{mit}\quad\sigma=\text{sign}=\pm 1\qquad\text{und}\quad a_i\in [0,B-1]
\end{equation*}

\subsubsection{Allgemeines}
\begin{itemize}
	\item gr"osste Maschinenzahl $B^{n+1}-B^m$
	\item gleichm"assige Verteilung
	\item jede im Maschinenbereich liegende Zahl l"asst sich durch eine Zahl $z^{\ast}$ approximieren mit $$|z-z^\ast|\leq\frac{1}{2}B^{-m}$$
	\item $(a\oplus B)\oplus c = a\oplus (b\oplus c)$
	\item $a\odot b \neq a\cdot b$
\end{itemize}

\paragraph{Vorteile}
\begin{itemize}
	\item effiziente Arithmetik
\end{itemize}

\paragraph{Nachteile}
\begin{itemize}
	\item kleiner Zahlenbereich
\end{itemize}


\subsection{Gleitpunktzahlen}
\begin{equation}
	z=\sigma\cdot m\cdot B^e\qquad e\in\mathbb{Z}\text{ und } e\in[e_{min},e_{max}]
\end{equation}
\begin{equation*}
	\text{mit}\qquad m=\sum\limits_{k=1}^n a_{-k}\cdot B^{-k}\qquad\text{mit}\quad a_k\in[0,B-1]
\end{equation*}

\subsubsection{Definition}
Eine Gleitpunktzahl ist normalisiert falls $$\frac{1}{B}\leq m < 1$$
d.h. die f"hrende Ziffer der Mantisse $\neq0$

\subsubsection{Allgemeines}
\begin{itemize}
	\item F"ur die Null existiert keine normalisierte Darstellung
	\item L"ucke bei Null: $[0,B^{e_{min}-1}]$ enth"alt keine Maschinenzahl
	\item Anzahl normalisierte Zahlen: $$2(B-1)\cdot B^{n-1}\cdot\Delta e = 2\cdot B^{n-1}(B-1)(e_{max}-e_{min}+1)$$
	\item Anzahl denomalisierte Zahlen: $$2\cdot B^{n-1}$$
\end{itemize}

\subsubsection{Definition}
\begin{equation}
	\mathbb{M}=M(B,n;e_0,e_1)=\left\{z|\sigma\left(\sum\limits_{k=1}^na_{-k}\cdot B^{-k}\right)\cdot B^{-k},e\in[e_0,e_1]\right\}
\end{equation}
\noindent $a\in M(B,n,e_0,e_1)$ heisst Maschinenzahl, darstellbare Zahl. \\
\noindent Charakteristika von $\mathbb{M}=M(B,n,e_0,e_1)$
\begin{itemize}
	\item Gr"osste darstellbare Zahl: $$(1-B^{-1}\cdot B^{e_1}$$
	\item betragsm"assig kleinste $$\text{normalisierte Zahl: }\frac{1}{B}\cdot B^{e_0}=B^{e_0-1}$$ $$\text{darstellbare Zahl gr"osser Null: } B^{.n}\cdot B^{e_0}=B^{e_0-n}$$
\end{itemize}

\noindent Allgemeines:
\begin{itemize}
	\item ungleichm"assig Verteilt
	\item Dichte nimmt bei wachsender Zahlengr"osse exponentiell ab
	\item Approximationsfehler bei grossen Zahlen entsprechend gr"osser
\end{itemize}

\section{Newton-Verfahren}

\subsection{Lineare Gleichungsysteme}
\begin{equation}
	x_{k+1})F(x_k)=x_k-\frac{f(x)}{f'(x)}
\end{equation}

\paragraph{Konvergenzgeschwindigkeit}
\begin{equation}
	\lim_{x\rightarrow a}F'(x)
\end{equation}

\paragraph{Konvergenzfaktor}
\begin{equation}
	F'(s)\qquad\text{mit }\quad s\text{ : Fixpunkt}
\end{equation}

\paragraph{Konvergenzordnung}
\begin{center}\begin{tabular}{l l l l}
	linear			& $F'(s)\neq 0$ 	& und	& $|F'(s)|<1$ \\
	quadratisch		& $F'(s)=0$			& und	& $F''(s)\neq 0$ \\
	kubisch			& $F''(s)=F's(s)=0$	& und	& $F'''\neq 0$
\end{tabular}\end{center}

\subsection{Nichtlineare Gleichungsysteme}
\begin{gather*}
	\underline{f}=\begin{pmatrix}f(x,y)\\g(x,y)\end{pmatrix} \\
	J_f=\begin{pmatrix}f_x & f_y \\ g_x & g_y\end{pmatrix}\qquad\text{Jacobi-Matrix} \\
	\underline{x}_{k+1}=\underline{x}_k-J_f^{(-1)}\cdot f\qquad\text{mit}\quad\underline{x}_0=\begin{pmatrix}x_0\\y_0\end{pmatrix} \\
	\text{Vereinfachung: }\quad\underline{h}=-J_f^{(-1)}\cdot f\quad\Longrightarrow\underline{x}_{k+1}=\underline{x}_k+\underline{h}
\end{gather*}

\section{Apriori Fehlerabsch"atzung}
\begin{align}
	\|x_n-x_k\| &\leq L^k<frac{1}{1-L}\cdot\|x_1-x_0\| \\
	k &\geq \frac{\ln\left(\epsilon\cdot(1-L)\cdot\|x_1-x_0\|^{-1}\right)}{\ln(L)}
\end{align}

\section{Newton-Cotes-Regeln}
Beispiel: Simpson
\begin{equation}
	I=\frac{h}{3}(y_0+4y_1+y_2)=\int\limits_a^b f(x)\,dx
\end{equation}
\begin{center}
	\begin{pspicture}(-0.5,-0.8)(5,3)
		\psline{->}(0,0)(5,0)\rput[br](5,0.2){$x$}
		\psline{->}(0,0)(0,3)\rput[tl](0.2,3){$y$}
		\psline[linestyle=dashed,linecolor=lightgray]{-}(0.2,2.0)(0.2,0.0)
		\psline[linestyle=dashed,linecolor=lightgray]{-}(2.5,1.0)(2.5,0.0)
		\psline[linestyle=dashed,linecolor=lightgray]{-}(4.5,2.5)(4.5,0.0)
		\psecurve[linecolor=red,showpoints=true]{-}(-0.3,1.5)(0.2,2.0)(2.5,1.0)(4.5,2.5)(5.0,3.0)
		\rput[t](0.2,-0.1){$x_0$}
		\rput[t](0.2,-0.5){$a$}
		\rput[t](2.5,-0.1){$x_1$}
		\rput[t](4.5,-0.1){$x_2$}
		\rput[t](4.5,-0.5){$b$}
		\rput[l](4.6,2.5){$f(x)$}
	\end{pspicture}
\end{center}
\begin{equation*}
	h=x_1-x_0
\end{equation*}

\section{Gauss-Quadratur}
\begin{equation}
	\int\limits_a^b f(x)\,dx\approx\sum\limits_{k=1}^n w_k\cdot f(x)
\end{equation}
\noindent Genauigkeitsgrad: 2n-1

\begin{align*}
	f\equiv 1 \qquad   & \int\limits_a^b\,dx = b-a = \sum\limits_{k=1}^n w_k \\
	f\equiv x \qquad   & \int\limits_a^b x\,dx = \frac{b^2}{2}-\frac{a^2}{2} = \sum\limits_{k=1}^n w_k\cdot x_k \\
	f\equiv x^2 \qquad & \int\limits_a^b x^2\,dx = \underbrace{\frac{b^3}{3}-\frac{a^3}{3}}_{\underline{b}} = \sum\limits_{k=1}^n w_k\cdot x_k^2 \\
\end{align*}

\noindent Vandermode-Matrix
\begin{equation}
	V\cdot\underline{w}=\underline{b}\qquad\Longrightarrow\quad\underline{w}=V^{-1}\cdot\underline{b}\qquad\text{falls}\quad\det(V)\neq0
\end{equation}

\section{Orthogonal-Polynome}
Skalarprodukt: $$(f,g)=\int\limits_a^b w_{(x)}\cdot f_{(x)}\cdot g_{(x)}\,dx$$ \\
Orthogonalpolynome $p_0,p_1,p_2,\ldots,\text{usw.}$
\begin{align*}
	p_0 &= 1 \\
	p_1 &= x+a \\
	p_2 &= x^2 + bx + c \\
	\vdots \\
	\text{usw.}
\end{align*}
\noindent Bedingung:
\begin{align*}
	(p_0,p_1) &= 0 \\
	(p_0,p_2) &= 0 \\
	(p_1,p_2) &= 0 \\
	\vdots \\
	\text{usw.}
\end{align*}
\noindent Alle Orthogonalpolynome m"ussen gegenseitig Skalarmultipliziert $0$ ergeben $\Longrightarrow$ Gleichungssystem in $a$, $b$, $c$, usw.

%
% EOF
%
