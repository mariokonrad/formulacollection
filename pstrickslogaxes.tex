\makeatletter

% D.G. - April 1998

% Define "log" parameter (log=all|x|y|none with default=none)
\def\psset@log#1{%
\pst@expandafter\psset@@ticks{#1}\@nil\psk@log%
% X axis labels (\psk@log = 0 or 1)
\ifnum\psk@log<\tw@\let\pshlabel\pshlabellog\fi
% Y axis labels (\psk@log = 0 or 2)
\ifodd\psk@log\else\let\psvlabel\psvlabellog\fi}
\psset@log{none}

% Default formats for log labels
\def\pshlabellog#1{$10^{#1}$}
\def\psvlabellog#1{$10^{#1}$}

% Define "ticklines" parameter (ticklines=all|x|y|none with default=none)
\def\psset@ticklines#1{\pst@expandafter\psset@@ticks{#1}\@nil\psk@ticklines}
\psset@ticklines{none}

% The parameter styles for the tick lines
% (default=arrows=-,linestyle=dotted,dotsep=5pt)
\def\TickLinesXStyle{}
\def\TickLinesYStyle{}

% Modified \psaxes@iv macro for log axis
\def\psaxes@iv(#1,#2)(#3,#4)(#5,#6){%
  \setbox\pst@hbox=\hbox\bgroup
    \use@par
    \pssetxlength\pst@dimg{#1}% o-x
    \pssetylength\pst@dimh{#2}% o-y
    \pssetxlength\pst@dima{#3}% bl-x
    \pssetylength\pst@dimb{#4}% bl-y
    \pssetxlength\pst@dimc{#5}% ur-x
    \pssetylength\pst@dimd{#6}% ur-y
% D.G. modification begin - Apr.  6, 1998
    % If minimum values are negative in log mode, we modify Ox
    % (respectively Oy) if this was not done by the user
    % X axis labels (\psk@log = 0 or 1)
    \ifnum\psk@log<\tw@
      \ifdim\psk@Ox pt=\z@
        \ifdim#3pt<\z@
          \pssetxlength\pst@dimg{#3}% o-x
          \psset@Ox{#3}%
        \fi
      \fi
    \fi
    % Y axis labels (\psk@log = 0 or 2)
    \ifodd\psk@log
    \else
      \ifdim#4pt<\z@
        \ifdim\psk@Oy pt=\z@
          \pssetylength\pst@dimh{#4}% o-y
          \psset@Oy{#4}%
        \fi
      \fi
    \fi
% D.G. modification end
% Whole thing will be translated to origin:
    \advance\pst@dima-\pst@dimg % Dist. from bl-x to o-x
    \advance\pst@dimb-\pst@dimh % Dist. from bl-y to o-y
    \advance\pst@dimc-\pst@dimg % Dist. from ur-x to o-x
    \advance\pst@dimd-\pst@dimh % Dist. from ur-y to o-y
% Make lines/arrows or frame:
    \@nameuse{psxs@\psk@axesstyle}%
% "\pslabelsep" should be from the edge of the axis.
    \advance\pslabelsep.5\pslinewidth
% Now the ticks and labels. Start by checking for "\multido".
% !!Need to fix this so that does nothing when there are 0 ticks.!!
    \begingroup
      \ifdim\pst@dimb=\z@\else\showoriginfalse\fi
      \ifnum\psk@dx=\z@
        \pst@dimg=\psk@Dx\psxunit
        \edef\psk@dx{\number\pst@dimg}%
      \fi
      \ifnum\psk@ticks<\tw@
        \ifnum\psk@tickstyle>\z@\else
          \advance\pslabelsep\psk@ticksize\p@
        \fi
      \fi
% D.G. modification begin - Apr.  8, 1998 - For vertical ticklines
%      \pst@hlabels\pst@dimc\psk@arrowB
%      \pst@hlabels\pst@dima\psk@arrowA
      \pst@hlabels{\pst@dimc}{\psk@arrowB}{#4}{#6}
      \pst@hlabels{\pst@dima}{\psk@arrowA}{#4}{#6}
% D.G. modification end
    \endgroup
    \begingroup
      \ifdim\pst@dima=\z@\else\showoriginfalse\fi
      \ifnum\psk@dy=\z@
         \pst@dimg=\psk@Dy\psyunit
         \edef\psk@dy{\number\pst@dimg}%
      \fi
      \ifodd\psk@ticks\else
        \ifnum\psk@tickstyle>\z@\else
          \advance\pslabelsep\psk@ticksize\p@
        \fi
      \fi
% D.G. modification begin - Apr.  8, 1998 - For horizontal ticklines
%      \pst@vlabels\pst@dimd\psk@arrowB
%      \pst@vlabels\pst@dimb\psk@arrowA
      \pst@vlabels{\pst@dimd}{\psk@arrowB}{#3}{#5}
      \pst@vlabels{\pst@dimb}{\psk@arrowA}{#3}{#5}
% D.G. modification end
    \endgroup
% Now close "\pst@hbox" (which is 0-dimensional), and put it at the origin.
  \egroup
  \pssetxlength\pst@dimg{#1}%
  \pssetylength\pst@dimh{#2}%
  \leavevmode\psput@cartesian\pst@hbox
  \ignorespaces}

% The origin is never the only label.
% D.G. modification begin - Apr.  8, 1998
% Modified \pst@hlabels macro for tick lines
%\def\pst@hlabels#1#2{%
\def\pst@hlabels#1#2#3#4{%
% D.G. modification end
  \ifdim#1=\z@\else
    \ifx#2\empty\else
      \advance#1\ifdim#1>\z@-\fi7\pslinewidth
    \fi
    \pst@cnta=#1\relax                % Distance (in sp) to end.
    \divide\pst@cnta\psk@dx\relax     % Number of ticks/labels
    \ifnum\pst@cnta=\z@\else
      \pst@dimb=\psk@dx sp            % Space between ticks.
      \ifnum\psk@ticks<\tw@
        \pst@ticks{0}{\pst@number\pst@dimb}{\the\pst@cnta}{\pst@dimd}%
      \fi
      \ifnum\psk@labels<\tw@ \pst@@hlabels\fi
      \showoriginfalse
% D.G. modification begin - Apr.  8, 1998 - For vertical ticklines
      \ifnum\psk@ticklines<\tw@
        \pst@cntb=\pst@cnta
        \ifnum\pst@cnta<\z@
          \multiply\pst@cntb\m@ne
        \fi
        \multips(\ifnum\pst@cnta<\z@-\fi\pst@dimb,0)%
                (\ifnum\pst@cnta<\z@-\fi\pst@dimb,0){\pst@cntb}{%
          \psset{arrows=-,linestyle=dotted,dotsep=5pt}
          \ifx\TickLinesYStyle\empty\else\TickLinesYStyle\fi
          \psline(0,#3)(0,#4)}
      \fi
% D.G. modification end
    \fi
  \fi}

% D.G. modification begin - Apr.  8, 1998
% Modified \pst@vlabels macro for tick lines
%\def\pst@vlabels#1#2{%
\def\pst@vlabels#1#2#3#4{%
% D.G. modification end
  \ifdim#1=\z@\else
    \ifx#2\empty\else
      \advance#1\ifdim#1>\z@-\fi7\pslinewidth
    \fi
    \pst@cnta=#1\relax                % Distance (in sp) to end.
    \divide\pst@cnta\psk@dy\relax     % Number of ticks/labels
    \ifnum\pst@cnta=\z@\else
      \pst@dima=\psk@dy sp            % Space between ticks.
      \ifodd\psk@ticks\else
        \pst@ticks{90}{\pst@number\pst@dima}{\the\pst@cnta}{-\pst@dimc}%
      \fi
      \ifodd\psk@labels\else\pst@@vlabels\fi
      \showoriginfalse
% D.G. modification begin - Apr.  8, 1998 - For horizontal ticklines
      \ifodd\psk@ticklines
      \else
        \pst@cntb=\pst@cnta
        \ifnum\pst@cnta<\z@
          \multiply\pst@cntb\m@ne
        \fi
        \multips(0,\ifnum\pst@cnta<\z@-\fi\pst@dima)%
                (0,\ifnum\pst@cnta<\z@-\fi\pst@dima){\pst@cntb}{%
          \psset{arrows=-,linestyle=dotted,dotsep=5pt}
          \ifx\TickLinesXStyle\empty\else\TickLinesXStyle\fi
          \psline(#3,0)(#4,0)}
      \fi
% D.G. modification end
    \fi
  \fi}

\makeatother

